\section{问题重述}\label{sec:peoblem_restatement}

本文研究的是路口信号灯实时智能控制方案研究,需要针对目前的红绿灯时长固定的机械转换方案进行改进。在车流量很大的路口,车辆排队进行等待的情况下,交通信号灯发挥着维持交通秩序的功能;但在车流稀少的情况,如果驾驶员沿东西方向行驶,此时南北方向无车驶过,如果没有交通信号灯,那么不需要减速即可通过路口,但在机械方案下,驾驶员需要减速停车等待半分钟以上的红灯,此时的交通信号灯显然阻碍了交通情况。此种阻碍,一方面增加了燃油量和碳排放,一方面影响了司机的心情和时间。

本文旨在,针对车流稀少的情况下,通过图像识别的方法,涉及实时控制信号灯的智能方案,最大程度提高路口的车辆的通行速度。

我们需要依次解决如下问题:(1)通过图像识别,对于提供的视频素材,通过观察信号灯的显示时间,视频记录的时间,汽车在图片中位置信息及其动态变化等等,提取出车辆进入、停车、驶离的种种信息。(2)通过数据处理,根据识别得出的信息,计算一下该路口的总体指标,例如车辆通过该路段的总流量以及平均速度。并且调查阻碍交通情况的总时间。(3)通过对路口优化方案的思考,以最大程度提供车辆通过路段的平均速度为目标,给出优化后的信号灯智能方案,并且根据新的方案,根据识别出的车辆进入信息,重新计算车辆的进出时间、进出距离等指标,计算出优化后的车辆的平均速度。

本题目的重心首先在于图像识别,图像可以识别出时间以及地理位置类信息,同时还需要进行图像坐标与世界坐标的变换求出图像中的距离信息。其次在于优化方案的思考。
