\section{问题重述}\label{sec:peoblem_restatement}

本文研究的是路口信号灯实时智能控制方案研究,需要针对目前的红绿灯时长固定的机械转换方案进行改进。在车流量很大的路口,车辆排队进行等待的情况下,交通信号灯发挥着维持交通秩序的功能;但在车流稀少的情况,如果驾驶员沿东西方向行驶,此时南北方向无车驶过,如果没有交通信号灯,那么不需要减速即可通过路口,但在机械方案下,驾驶员需要减速停车等待半分钟以上的红灯,此时的交通信号灯显然阻碍了交通情况。此种阻碍,一方面增加了燃油量和碳排放,一方面影响了司机的心情和时间。

本文旨在,针对车流稀少的情况下,通过图像识别的方法,获取时间、车辆位置等信息,通过数据得到该路口诸如车流量、平均速度等基本信息,通过对交通灯阻碍交通的情况进行调查和分析,基于现状给出实时控制信号灯的智能方案,最大程度提高路口的车辆的通行速度。

我们需要依次解决如下问题:(1)通过图像识别,对于提供的视频素材,通过观察信号灯的显示时间,视频记录的时间,汽车在图片中位置信息及其动态变化等等,提取出车辆进入、停车、驶离的种种信息。(2)通过数据处理,根据识别得出的信息,计算一下该路口的总体指标,例如车辆通过该路段的总流量以及平均速度。给出阻碍行为的具体定义和算法,并且计算阻碍交通情况的总时间。(3)通过对路口优化方案的思考,以最大程度提高车辆通过路段的平均速度为目标,给出优化后的信号灯智能方案,并且根据新的方案,根据识别出的车辆进入信息,重新计算车辆的进出时间、进出距离等指标,计算出优化后的车辆的平均速度。

本题目的重心首先在于图像识别,图像可以识别出时间以及地理位置类信息,同时还要对图像识别的准确性进行优化,再进行图像坐标与世界坐标的变换求出图像中的距离信息。其次在于阻碍行为将一个相位的车辆数据与其他相位的数据联系在一起,在不同的相位考虑相互的阻碍,定义和算法都略难梳理。最后信号灯配时问题,适配车辆稀少情况下的配时算法研究的较少,故对算法的修正也是重点。
