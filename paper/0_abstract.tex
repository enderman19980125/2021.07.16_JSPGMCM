\begin{center}
    摘\qquad 要
\end{center}
\qquad 对路口信号灯实时智能控制方案研究分为了三部分,首先需要根据视频文件,进行图像识别,导出11:00至11:30的路口车辆的时间、大小,位置等信息。问题一需要统计视频中所有出现车辆的16项数据,需要运用计算机视觉领域的算法,属于图像识别和目标检测任务。我们先将视频转换为逐帧的图片序列。首先我们需要识别图像中的监控时间和红绿灯秒数,由于每个路口的监控时刻和红绿灯秒数在图像中的位置固定,因此我们人为地框定需要识别数字的区域。我们使用基于opencv实现的传统数字图像处理算法,结合基于 MNIST 预训练的 Resnet18 神经网络共同完成数字识别任务,当两种算法结果不一致时再人工纠正。其次我们检测图像中出现的车辆,我们使用在 coco 数据集上预训练的 tiny-YOLOv3 神经网络逐帧检测每张图像中的车辆,输出每辆车的 Bounding Box,并使用链接算法将帧与帧之间车辆的 Bounding Box 链接形成车辆轨迹。最后,我们使用投影算法测算图像中对应点位的实际距离。我们建立了若干蒙版(车道蒙版、路口蒙版、距离蒙版)用于识别 Bounding Box 所处的位置和离停车线的距离。
在求得第一问数据的基础上,我们得出了路口半小时内总流量为732辆,东西南北口分别为:217、202、120、193.平均速度为0.8290m/s.

交通信号灯先进行周期和相位划分,南北方向直行、南北方向左转、东西方向直行、东西方向左转分别为一二三四相位,总周期长为113秒。我们认为每个周期相互独立,分别求取周期内的阻碍时间。对于阻碍行为的定义中,我们只考虑相邻时间区间的阻碍行为。只存在直行阻碍同向的左转,以及左转阻碍异向的直行,两种情况,在实际考虑相位时,有四种含方向的阻碍类型。
阻碍时间的具体计算思想为:以双向车道中最后一辆车驶离停车线的绿灯显示时间为该相位车道阻碍另一相位的阻碍时间,与另一相位在最后一辆车辆驶离时刻之前已在停车等待的停车数量乘积结果为该相位的阻碍时长,再将四个相位结果相加,最终的阻碍时间为:
第三部分是信号灯配时优化问题。

第三问主要是信号灯时配的优化问题。首先对现有数据进行预处理,计算出各周期不同相位实际流通量,并依据该信息将总时间段按照不同相位特征分为三个子时间段。以减少平均延误、进入速度损失、增大通行能力为目的,构建目标函数。优化方法采用传统的Webster算法和遗传算法,求得最优周期和各相位有效绿灯时长。在此基础上再次输出进入车辆信息进行流通模拟,得到更新后的表格,并和原有信息进行对比分析。结果发现优化方案均能提高通过路口的平均速度,遗传算法所得结果效果更好。

关键字:图像识别;目标检测;Webster算法;遗传算法;交比